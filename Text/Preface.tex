\thispagestyle{empty}
\cleardoublepage
\thispagestyle{empty}
\begin{titlepage}
\begin{center}
%{\rugfnt A}\\[.3cm]
\vspace{5cm}
\HRule \\[1cm]
\ifanalysis{\Huge  Calculus and Analysis}\\[.7cm]\fi
\ifcalculus{\Huge  Calculus \ifDE with Differential Equations\fi}\\[.7cm]\fi
\HRule \\[3cm]
\end{center}

\begin{center}
{\Large Prof.\ dr.\ ir.\ Jan M.\ Baetens\\
\Large M.Sc.\ Elien Van de Walle\\
\Large M.Sc.\ An Schelfaut}\\[9cm]
%\begin{flushleft}
%\includegraphics[width=8cm]{logo_UGent_NL_RGB_kleur}
%\end{flushleft}
\vfill
\end{center}
\end{titlepage}
\thispagestyle{empty}
\vspace*{4cm}
The cover photo represents a hyperbolic paraboloid whose standard equation is given by
$$
z=\dfrac{x^2}{a^2}-\dfrac{y^2}{b^2}\,.
$$
\\[14cm]
\textcopyright\, 2022 Jan M.\ Baetens.\\
 Licensed to the public under Creative Commons Attribution-Noncommercial 4.0 International Public License. The course is largely based on  chapters from \textit{Precalculus} by Carl Stitz and Jeff Zeager, chapters from \textit{APEX Calculus} by Gregory Hartman et al.\  and own material. 
 	\begin{flushleft}
			\includegraphics[width=0.2\textwidth]{cc}
	\end{flushleft}

\chapter*{Preface}
\addcontentsline{toc}{chapter}{Preface}
The purpose of this course is to present mathematics as the science of deductive reasoning and not as the art of manipulation. Unfortunately, many students feel mathematics is incomprehensible and is riddled with complex and abstract jargon. Our goal is to impose a lasting understanding of and appreciation for calculus on the student.  Our course is intended to give the student an understanding of what calculus is truly about. It does not take more intelligence than that of a parrot to be able to go through a list of theorems and equations; but only when one understands their origins can one correctly and confidently apply them in the real world. 

The over-emphasis on the calculator and foremostly the computer is  definitely a point of confusion for the student. The computer is only a time-saving machine whose usefulness depends on the knowledge of the user. We do admit the computer is a remarkable machine, and we will make use of it whenever appropriate, yet it is this fascination that gives students a false sense of what they are doing. The confidence gained from all the correct answers leads to an inseparable dependence where the student is absolutely helpless without it.


Throughout the textbook we constantly refer to science and engineering. The purpose of this is to show how the scientific method applies to all disciplines and to understand that mathematics is an expression of one's observations and hypothesis. For that reason, several examples and exercises were chosen because of their relevance in reality, such that the reader can get a good feel of why and how this course is so important for future engineers. Note that because of its engineering viewpoint, we always indicate the dimensions of the used base quantities, being mass [M], time [T], temperature [$\Theta$] and length [L]. Throughout this course the icon  \includegraphics[width=0.025\textwidth]{youtube} in the margin indicates that there's a supporting You Tube video available. The QR-code below  takes you directly to the appropriate You Tube video online. At the end of every chapter one can find an extensive list of exercises linked to the topics discussed in the corresponding chapter. 

Even though much time and efforts have been spent in compiling this text, it cannot be free of errors, and the authors would be grateful if these would be reported to them so that the quality of this text can be improved even further.

It goes without saying that many people have contributed to this course in addition to its authors, namely, Demir Ali K\"ose, Janos Coquyt, Tinne De Boeck, Diego De Gusem, Lander De Visscher, Wannes Dewulf, Jeroen Galle, Jelle Hustinx, Hanna Jaspaert, Linde Lambrecht, Robin Simoens, Ward Van Belle, Caitlin Vanden Bussche, Victor Vanthilt and Hilder Vernieuwe. 

Finally, we are grateful that Ben Orlin, author of the book \textit{Math with Bad Drawings} and the blog \url{https://mathwithbaddrawings.com/}, granted us permission to include  his cartoons at the end of some of the  chapters. 
\vspace*{2cm}


\begin{flushleft}
Ghent, September 10, 2020\\[0.5cm]
The authors
\end{flushleft}