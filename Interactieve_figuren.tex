Ik ben zopas eens door jullie interactieve figuren op ufora gegaan en heb volgende suggesties voor figuren voor het remdiëringstraject:
Bewerkingen met verzamelingen
Functietransformaties en inverse functies 
Som van 2 vectoren
Afgeleide als een functie

Figuren die nog niet bestaan, maar eventueel wel nuttig kunnen zijn:
Goniometrische getallen
Afgeleide van een functie en raaklijn
Bestaan van limieten (vb. sin(1/x))
Transcendente functies: sinusoide, exp versus log, cos en arccos

Voorstel An:
Kegelsneden (vanuit algemene vorm?)
Figuren in 3D (rechten, vlakken, kwadrieken)
Numerieke nulpuntsbepaling?
Numerieke integratie (trapeziummethode,...) of algemeen het idee van dunne rechthoekjes benaderen oppervlakte goed?
Taylorontwikkelingen van gekende functie (sin, exp,...) en tonen dat meer termen een betere benadering geven
Iets over dvergentie, rotatie, flux,... ?

-> Kijken of er nog interessante filmpjes bestaan van Khan Academy?
-> De site https://wiskunde-interactief.be/ kan eventueel interessant zijn voor remediëring of vakantiecursus.